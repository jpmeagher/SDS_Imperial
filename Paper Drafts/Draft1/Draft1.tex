%%%%%%%%%%%%%%%%%%%%%%%%%%%%%%%%%%%%%%%%%%%%%%%%%%%%%%%%%%%%%%%%%%%%%%%%%%
%% Review Volume (last updated on 20-4-2015)                            %%
%% Trim Size: 9in x 6in                                                 %%
%% Text Area: 7.35in (include runningheads) x 4.5in                     %%
%% Main Text: 10 on 13pt                                                %%
%% For support: Yolande Koh, <ykoh@wspc.com.sg>                         %%
%%              D. Rajesh Babu, <rajesh@wspc.com.sg>                    %%
%%%%%%%%%%%%%%%%%%%%%%%%%%%%%%%%%%%%%%%%%%%%%%%%%%%%%%%%%%%%%%%%%%%%%%%%%%
%%
\documentclass[wsdraft]{ws-rv9x6} % to draw border line around text area
%\documentclass{ws-rv9x6}
\usepackage{subfigure}   % required only when side-by-side / subfigures are used
\usepackage{ws-rv-thm}   % comment this line when `amsthm / theorem / ntheorem` package is used
\usepackage{ws-rv-van}   % numbered citation & references (default)
%\usepackage{ws-index}   % to produce multiple indexes
\makeindex
%\newindex{aindx}{adx}{and}{Author Index}       %author index
%\renewindex{default}{idx}{ind}{Subject Index}  %subject index

\begin{document}

\chapter[Ancestral Reconstruction of Bat Echolocation Calls]{Ancestral Reconstruction of Bat Echolocation Calls}\label{ra_ch1}

\author[J.P. Meagher et al.]{J.P. Meagher\footnote{Author footnote.}}
%\index[aindx]{Author, F.} % or \aindx{Author, F.}
%\index[aindx]{Author, S.} % or \aindx{Author, S.}

\address{Department of Statistics,\\
University of Warwick, \\
J.Meagher@Warwick.ac.uk\footnote{Affiliation footnote.}}

\begin{abstract}
 Something something bats something evolution something ancestral reconstruction something magic.
\end{abstract}
%\markright{Customized Running Head for Odd Page} % default is Chapter Title.
\body

%\tableofcontents
\section{Introduction}

Advances in technology allowing the precise quantification and storage of information about the world around us continues to drive the emergence of Data Science as a discipline distinct from both Statistics and Computer Science.

Bioacoustics is one area of research generating vast quantities of data which also captures the imagination of the public, as evidenced by successful citizen science initiatives. \cite{allen2006citizen} \cite{pettorelli2013indicator} Bioacoustic techniques for biodiversity monitoring \cite{stathopoulos2017bat} \cite{damoulas2010bayesian} have the potential to make real policy impacts, particularly with regard to sustainable economic development and nature conservation.

In the acoustic monitoring of biodiversity, bats (order \textit{Chiroptera}) are of particular interest. Bats have been identified as ideal bioindicators for monitoring climate change and habitat quality,\cite{jones2009carpe} largely because bats broadcast information about themselves into their environment in the form of echolocation calls.\cite{jones2005echolocation} The development of automatic acoustic monitoring algorithms for classifying species of bats \cite{stathopoulos2017bat} \cite{walters2012continental} means that large scale, non-invasive monitoring is becoming possible. 

While monitoring bat populations provides useful information, understanding the root causes and effects of what is observed requires that the natural history of extant bat species is also well understood. Traits, such as call structure or body size, exhibited by particular bat species are linked to the bats interactions with its environment. \cite{aldridge1987morphology} Existing fossil records are of limited use in inferring the traits exhibited by ancestral bats, particularly with respect to echolocation calls. The reconstruction of ancestral traits relies heavily on the comparative analysis\cite{felsenstein2004inferring} of extant bat species. Thus, statistical data science techniques may be particularly useful for inferring the evolutionary dynamics and reconstructing ancestral states of echolocation in bats.

Previous studies of bat echolocation calls for both classification \cite{walters2012continental} and ancestral reconstruction \cite{collen2012evolution} examined features of the call extracted from the call spectrogram. These, somewhat arbitrary, call features relied on significant domain knowledge to ensure they were sensibly selected and used. More recently however, general techniques for the classification of acoustic signals have been developed. \cite{stathopoulos2014bat} \cite{damoulas2010bayesian} These methods do not require, but can be augmented by, domain knowledge. General techniques for ancestral reconstruction of function-valued traits, such as speech sounds or echolocation calls, have been proposed \cite{group2012phylogenetic}. The study of bat echolocation calls offers an opportunity to examine the efficacy of these techniques.

A function-valued trait is measured along some continuous scale, usually time, and can then be modelled as a continuous mathematical function using techniques for functional data analysis. \cite{ramsay2006functional} Jones \& Moriarty \cite{jones2013evolutionary} developed a method which extends Gaussian Process Regression \cite{rasmussen2006gaussian} to model the evolution of function-valued traits over a phylogeny. The model facilitates the implementation of two popular models for continuous character state evolution,\cite{joy2016ancestral} the Brownian Motion and Ornstein-Uhlenbeck models.\cite{lande1976natural} A full demonstration of ancestral reconstruction for synthetic data using the method was presented by Hajipantelis et al.\cite{hadjipantelis2013function}

This general approach to evolutionary inference for function-valued traits is implemented here for a set of bat echolocation calls. Our goal in doing so is twofold. These techniques had previously been considered in the context of modelling the evolution of human speech sounds in language. \cite{group2012phylogenetic} It is hoped that by applying these methods in the simpler context of the evolution of bat echolocation calls that progress can be made towards resolving methodological problems. For example, how do we extend these methods to more realistic models of evolution?  

We are also interested in what specifically these models tell us about bats and the evolutionary dynamics driving the development of echolocation. What impact might these results have on our understanding of ancestral bats and their behaviour?

This paper presents the early stages of our research and some preliminary results.

\section{Functional Representation of Bat Echolocation Calls}

A functional data object is generated when repeated measurements of some process are taken along a continuous scale, such as time. \cite{ramsay2006functional} 
These measurements can be thought of as representing points on a curve that varies gradually and continuously. In the context of phylogenetics, these functional data objects are called function-valued traits. \cite{meyer2005up}

Denote the \(m^{th}\) call recording of the \(l^{th}\) individual bat of species \(S\) by \(\{\tilde{x}_{lm}^S(n) : n = 0, \dots, N_{lm}^S - 1\}\). 
Thus \(\{\tilde{x}_{lm}^S(\cdot)\}\) is a noisy realisation of \(x^S(\cdot)\), the echolocation call generating process for species \(S\), observed at the time points given by \(\frac{n}{f_S}\), where \(f_S\) is the process sampling rate in samples per second (Hz).

Call recordings themselves are in fact functional data objects, however modelling the phylogenetic relationships between \(\{\tilde{x}_{lm}^S(\cdot)\}\) and \(\{\tilde{x}_{l'm'}^{S'}(\cdot)\}\) directly implies that the processes are comparable at point \(n\). 
This is not the case for acoustic signals, which are sinusoidal and can vary in time without significantly altering the information carried. Thus some alternative functional representation of the signal is required. 

The discrete Fourier transform of the signal \(\{\tilde{x}_{l'm'}^{S'}(\cdot)\}\) is given by\[\tilde{X}_{lm}^S(k) = \sum_{n = 0}^{N_{lm}^S - 1} \tilde{x}_{lm}^{S}(n) e^{-i2\pi kn / N_{lm}^S }.\] 
The energy spectral density of this signal is the magnitude of the Fourier transform and so the log energy spectral density per second (in decibel) is estimated by \[\tilde{\mathcal{E}}_{lm}^S(k) = 10 \log_{10} \left( \frac{|\tilde{X}_{lm}^S(k)|f_s}{N_{lm}^S}\right).\]
The log energy spectral density estimate for the signal \(\tilde{x}_{lm}^S(\cdot)\), \(\tilde{\mathcal{E}}_{lm}^S(\cdot)\) is now considered to be a noisy estimate of the log energy spectral density for species \(S\), denoted \(\mathcal{E}^S(\cdot)\). The log energy spectral density is a periodic function of frequency which describes the energy of a signal at each frequency on the interval \(F = [0, \frac{f_S}{2}]\). \(\tilde{\mathcal{E}}_{lm}^S(\cdot)\) has been mapped to the decibel scale and also scaled according to the length, in time, of \(\tilde{x}_{lm}^S(\cdot)\).\cite{antoniou2006digital} We have now mapped each echolocation call on the same scale and \(\tilde{\mathcal{E}}_{lm}^S(\cdot)\) is now comparable to \(\tilde{\mathcal{E}}_{l'm'}^{S'}(f)\) at frequency \(f\). In order for these representations to be considered as function-valued traits however, \(\tilde{\mathcal{E}}_{lm}^S(\cdot)\) must be smoothed such the call representation in the frequency domain varies gradually and continuously.

The smoothed log energy spectral density is estimated by smoothing splines where \[
\mathcal{E}_{lm}^S(f) = \arg \min_{\mathcal{E}_{lm}^S(\cdot)} \int_{0}^{\frac{f_S}{2}}  \{ \tilde{\mathcal{E}}_{lm}^S(\cdot) - \mathcal{E}_{lm}^S(f) \}^2 df + \lambda \int_{0}^{\frac{f_S}{2}} \{ {\mathcal{E}_{lm}^{S}}'' (f)\}^2 df.
\]
The smoothing parameter \(\lambda\) is chosen using a generalised cross-validation procedure.\cite{friedman2001elements} \cite{citeR}

We now have a functional representation of each bats echolocation call where the pairs of observations \(\{f, \mathcal{E}_{lm}^S(f)\}\) and \(\{f, \mathcal{E}_{l'm'}^{S'}(f)\}\) are directly comparable. The function-valued traits can now be modelled for evolutionary inference.

\section{Phylogenetic Gaussian Process Regression for Bat Echolocation Calls} 

\subsection{Gaussian Process Regression on Phylogenies}

The key innovation of Jones \& Moriarty\cite{jones2013evolutionary} in extending Gaussian Process Regression \cite{rasmussen2006gaussian} for use in evolutionary inference, was to replace the linear measure of distance between observations with a phylogenetic tree, denoted \(\mathbf{T}\). When this condition is imposed each of our observations \(\mathcal{E}^{S}(f)\) correspond to a point \((f, \mathbf{t})\) on the frequency-phylogeny \(F \times \mathbf{T}\). It is then by constructing a phylogenetic covariance function \(\Sigma_{\mathbf{T}}\left(\mathcal{E}^{S}(\cdot), \mathcal{E}^{S'}(\cdot)\right)\) that evolutionary inference can be performed.

Deriving a tractable form of the phylogenetic covariance function requires some simplifying assumptions. Firstly, it is assumed that conditional on their common ancestors in the phylogenetic tree \(\mathbf{T}\), any two traits are statistically independent. 

The second assumption is that the statistical relationship between a trait and any of it's descendants in \(\mathbf{T}\) is independent of the topology of \(\mathbf{T}\). That is to say that the underlying process driving evolutionary changes is identical along all individual branches of the tree. We call this underlying process along each branch the marginal process. The marginal process depends on the date of \(\mathbf{t}\), the distance between a point \(\mathbf{t} \in \mathbf{T}\) and the root of \(\mathbf{T}\), denoted \(t\). 

Finally, we assume that the covariance function of the marginal process is separable over evolutionary time and the function-valued trait space. Thus, by defining the frequency only covariance function \(K(f,f')\) and the time only covariance function \(k(t,t')\) the covariance function of the marginal process is
\[\Sigma\left((f,t), (f',t')\right) = K(f, f') k(t,t')\]

Under these conditions, Jones \& Moriarty \cite{jones2013evolutionary} show that the phylogenetic covariance function is also separable, that is\[\Sigma_{\mathbf{T}}\left((f,\mathbf{t}), (f',\mathbf{t}')\right) = K(f, f') k_{\mathbf{T}}(\mathbf{t},\mathbf{t}').\] 

It is also shown that for a phylogenetic Gaussian Process with this covariance function, \(Y\), and a degenerate Mercer kernel, \(K(\cdot, \cdot)\), there exists a set of \(n\) deterministic basis functions \(\phi_i: F \to \mathbf{R}\) and univariate Gaussian processes \(X_i\) for \(i = 1,\dots, n\) such that 
\[
g(f, \mathbf{t}) = \sum_{i = 1}^{n} \phi_i(f) X_i(\mathbf{t})
\] 
has the same distribution as \(Y\). 

Thus, given an appropriate set of basis functions, \(\phi_{\mathcal{E}} = [\phi^{\mathcal{E}}_1(\cdot), \dots, \phi^{\mathcal{E}}_n(\cdot)]\), and Gaussian Processes, \(X_{\mathcal{E}} = [X_1^{\mathcal{E}}(\cdot), \dots\ X_n^{\mathcal{E}}(\cdot)]\), the set of observations of the echolocation function-valued trait can be expressed in matrix notation as 
\[
\mathcal{E} = X_{\mathcal{E}} \phi_{\mathcal{E}}^{\mathsf{T}},
\]
where \( X_{\mathcal{E}}\) is the matrix of mixing coefficients of the fixed basis functions determining the function-valued trait. The values of \( X_{\mathcal{E}}\) are modelled as evolving by univariate phylogenetic Gaussian Processes.

\subsection{Deterministic Basis Functions}

Applying this model to observed traits requires that \(\phi\) be estimated somehow. Hajipantelis et al. \cite{hadjipantelis2013function} addressed this problem. The model outlined above implicitly assumes that the rows of \(X_{\mathcal{E}}\) are independent. This in turn implies that each of the basis, \(\phi_i(\cdot)\), evolved independently of one another. \(\hat{\phi}\), the estimate for \(\phi\), must reflect this. 

A Functional Principal Components Analysis \cite{ramsay2006functional} of the traits would return a set of orthogonal basis functions. This dimension reduction technique allows the selection of \(n\) basis functions which describe some proportion of the variation in the sample. However, this implicitly assumes that the basis functions are also Gaussian, a strong assumption which may not be realistic.

A less stringent condition is to assumes only that the basis functions are independent. Such a set of components can be found by an Independent Components Analysis. Blasche \& Wiskott\cite{blaschke2004cubica} present a method for deriving Independent Components. This two step procedure first implements a Principal Components Analysis to estimate the effective dimensionality of the dataset, before passing the effective dimensions to the CuBICA algorithm. This algorithm then rotates these effective dimensions until the third and fourth cumulants have also been diagonalised, which produces approximately independent basis functions.

\subsection{The Phylogenetic Ornstein-Uhlenbeck Process for Evolutionary Inference}

The relationships between the mixing coefficients, \(X_{\mathcal{E}}\), are modelled by a phylogenetic Gaussian Process, which must be defined. The Ornstein-Uhlenbeck process offers a popular method of modelling stabilising selection in comparative studies.\cite{hansen1997stabilizing} \cite{collen2012evolution} \cite{hadjipantelis2013function} Here, each independent phylogenetic Gaussian Process, \(X_i(\cdot)\), is modelled as an Ornstein-Uhlenbeck process.

The phylogenetic Ornstein-Uhlenbeck process is defined by the kernel
\[
k_{\mathbf{T}}^i(\mathbf{t}, \mathbf{t}') = (\sigma_p^i)^2 \exp \left( \frac{-d_{\mathbf{T}}(\mathbf{t}, \mathbf{t}')}{\ell^i} \right) + (\sigma_n^i)^2 \delta_{\mathbf{t}, \mathbf{t}'}
\] 
where \(\delta\) is the Kronecker delta, \(d_{\mathbf{T}}(\mathbf{t}, \mathbf{t}')\) is the cophenetic distance between the points \(\mathbf{t}\) and \(\mathbf{t}'\) on the phylogeny \(\mathbf{T}\), and \(\mathbf{\theta}^i = [\sigma_p^i, \ell^i, \sigma_n^i]^{\mathsf{T}}\) is the vector of hyperparameters for the process \(X_i(\cdot)\).

The full phylogenetic covariance function is then
\[
\Sigma_{\mathbf{T}}((f, \mathbf{t}), (f', \mathbf{t}')) = \sum_{i = 1}^{n}  k_{\mathbf{T}}^i(\mathbf{t}, \mathbf{t}') \phi_i^{\mathcal{E}}(f) \phi_i^{\mathcal{E}}(f')
\]
and the log likelihood associated with the model is
\[
\ell(\mathcal{E} | \theta) = -\frac{1}{2} \sum_{i = 1}^{n} \left( X_i(\cdot)^{\mathsf{T}} k_{\mathbf{T}}^i(\cdot, \cdot)^{-1}  X_i(\cdot) + \log( \det (k_{\mathbf{T}}^i(\cdot, \cdot))) + |X_i(\cdot)| \log 2\pi   \right),
\]
where \(\theta = [\theta^1, \dots , \theta^n]\), and \(|X_i(\cdot)|\) is the Euclidean length of the vector of realisations of \(X_i(\cdot)\). 

Model selection can be performed by a type II maximum likelihood estimation procedure which maximises the likelihood of the sample with respect to \(\theta\).

The model hyperparameters have intuitive interpretations. The variance of observations in the sample is \(\sigma_p + \sigma_n\), where \(\sigma_p\) is the phylogenetic noise, and \(\sigma_n\) is the non-phylogenetic noise. \(\sigma_p\) is the proportion of the variance within the sample which due to phylogenetic relationships, while \(\sigma_n\) accounts for other sources of variation. The length-scale parameter, \(\ell\) then indicates the strength of the correlation between traits at various points on the phylogeny, where strong correlations are given by a large \(\ell\).

Finally, ancestral reconstruction of the function-valued trait at some unobserved point in the phylogeny, \(\mathcal{E}^*\), is given by its posterior distribution 
\[
p(\mathcal{E}^* | \mathcal{E})  \sim \mathcal{N}(A, B)
\]
where 
\[
A = \sum_{i=1}^{n} k_{\mathbf{T}}^i(\mathbf{t}^*, \cdot) k_{\mathbf{T}}^i(\cdot, \cdot)
^{-1} X_i^{\mathcal{E}} (\cdot) \phi_i^{\mathcal{E}}
\]
and
\[
B = \sum_{i=1}^{n}\left( k_{\mathbf{T}}^i(\mathbf{t}^*, \mathbf{t}^*) -  k_{\mathbf{T}}^i(\mathbf{t}^*, \cdot) k_{\mathbf{T}}^i(\cdot, \cdot)
^{-1} k_{\mathbf{T}}^i(\mathbf{t}^*, \cdot)^{\mathsf{T}} \right) \phi_i^{\mathcal{E}}
\]

Thus, all the tools required to perform an ancestral reconstruction of the function-valued trait have been defined. 

\section{Results}

\subsection{Data Description}

The post processed echolocation call data accompanying Stathopoulos et al. \cite{stathopoulos2017bat} was used in this analysis. Echolocation calls were recorded across north and central Mexico with a Pettersson 1000x bat detector (Pettersson Elektronik AB, Uppsala, Sweden). Live trapped bats were measured and identified to species level using field keys.\cite{ceballos2005mamiferos} \cite{medellin2sanchez} Bats were recorded either while released from the hand or while tied to a zip line. The bat detector was set to record calls manually in real time, full spectrum at 500 kHz. 
In total the dataset consists of 22 species in five families, 449 individual bats and 1816 individual echolocation call recordings. 

Collen's \cite{collen2012evolution} Bat super-tree provided the basis for the phylogenetic tree of the recorded bat species, \(\mathbf{T}\). 

\begin{table}[ht]
	\tbl{Echolocation Call Dataset Statistics}
	{\begin{tabular}{@{}lccc@{}} \toprule
		Species & Key & Individuals & Calls \\ \colrule
		Family: Emballonuridae &&& \\
		1 \textit{Balantiopteryx plicata} & Bapl & 16 & 100 \\
		\\
		Family: Molossidae &&& \\
		2 \textit{Nyctinomops femorosaccus} & Nyfe & 16 & 100 \\
		3 \textit{Tadarida brasiliensis} & Tabr & 49 & 100  \\
		\\
		Family: Vespertilionidae &&& \\
		4 \textit{Antrozous pallidus} & Anpa & 58 & 100 \\
		5 \textit{Eptesicus fuscus} & Epfu & 74 & 100 \\
		6 \textit{Idionycteris phyllotis} & Idph & 6 & 100 \\
		7 \textit{Lasiurus blossevillii} & Labl & 10 & 90 \\
		8 \textit{Lasiurus cinereus} & Laci & 5 & 42 \\
		9 \textit{Lasiurus xanthinus} & Laxa & 8 & 100 \\
		10 \textit{Myotis volans} & Myvo & 8 & 100 \\
		11 \textit{Myotis yumanensis} & Myyu & 5 & 89 \\
		12 \textit{Pipistrellus hesperus} & Pihe & 85 & 100 \\
		\\
		Family: Mormoopidae &&& \\
		13 \textit{Mormoops megalophylla} & Mome & 10 & 100 \\
		14 \textit{Pteronotus davyi} & Ptda & 8 & 100 \\
		15 \textit{Pteronotus parnellii} & Ptpa & 23 & 100 \\
		16 \textit{Pteronotus personatus} & Ptpe & 7 & 51 \\
		\\
		Family:Phyllostomidae &&& \\
		17 \textit{Artibeus jamaicensis} & Arja & 11 & 82 \\
		18 \textit{Desmodus rotundus} & Dero & 6 & 38 \\
		19 \textit{Leptonycteris yerbabuenae} & Leye & 26 & 100 \\
		20 \textit{Macrotus californicus} & Maca & 6 & 53 \\
		21 \textit{Sturnira ludovici} & Stlu & 8 & 51 \\
		22 \textit{Sturnira lilium} & Stli & 4 & 20 \\
		\botrule
		\end{tabular}
	}
	\label{tab::dataset}
\end{table}

\subsection{Simulation Study}

In order to demonstrate the efficacy of hyperparameter estimation for a phylogenetic Ornstein-Uhlenbeck process over the phylogeny \(\mathbf{T}\), a short simulation study was carried out. For this study, samples of a univariate phylogenetic Ornstein-Uhlenbeck processes with hyperparameters \(\theta= [1,50,1]^{\mathsf{T}}\) were generated. Type II maximum likelihood point estimates of the hyperparameters, were made for each sample. The samples varied in the number of observations taken at the leaf nodes of \(\mathbf{T}\), denoted \(n\), corresponding to the number of observations taken per species. This resulted in a full sample of size \(n \times 22\). The experiment was repeated until there were 1000 samples each for \(n \in \{1,2,4,8\}\). The distribution of the type II maximum likelihood hyperparameter estimates was then examined.

Removes estimates where \(\ell > 1000\) to stabilise distribution.

\begin{table}[ht]
	\tbl{\(\hat{\theta}\) Distribution}
	{
		\begin{tabular}{@{}cccc@{}} \toprule
			\(n\)  & \(\sigma_p \) & \(\ell\) & \(\sigma_n\) \\
			\colrule
			1 & 1.09 (0.47) & \(10^{14}\) (\(10^{15}\)) & 0.57 (0.54) \\
			2 & 0.97 (0.29) & \(10^{13}\) (\(10^{14}\)) & 0.99 (0.15) \\
			4 & 0.97 (0.25) & 63.66 (136.96) & 1.00 (0.09) \\
			8 & 0.99 (0.24) & 56.21 (48.24) & 1.00 (0.06) \\			
			\botrule
		\end{tabular}
	}
	\label{tab::simulation_option2}
\end{table}

\begin{table}[ht]
	\tbl{\(\hat{\theta}\) Distribution}
	{
		\begin{tabular}{@{}cccc@{}} \toprule
			\(n\)  & \(\sigma_p \) & \(\ell\) & \(\sigma_n\) \\
			\colrule
			1 & 0.46 (0.21) & \(10^{14}\) (\(10^{15}\)) & 0.07 (0.08) \\
			2 &  0.45 (0.20) & 259.18 (309.02) & 0.10 (0.02) \\
			4 & 0.45 (0.20) & 279.50 (363.12) & 0.10 (0.01) \\
			8 & 0.47 (0.21) & 291.61 (343.14) & 0.10 (0.01) \\			
			\botrule
		\end{tabular}
	}
	\label{tab::simulation_option3}
\end{table}

\begin{table}[ht]
	\tbl{\(\hat{\theta}\) Distribution}
	{
		\begin{tabular}{@{}cccc@{}} \toprule
			& \(\sigma_p \) & \(\ell\) & \(\sigma_n\) \\
			\colrule
			Data Generating values & 0.50 & 250 & 0.10 \\
			Full Sample Estimate &  0.41 & 156.75 & 0.10 \\
			Resampled Estimates & 0.42 (0.01) & 161.06 (27.11) & 0.10 (0.01) \\		
			\botrule
		\end{tabular}
	}
	\label{tab::simulation_option4}
\end{table}


The results of this simulation study suggest that for any given sample \(\hat{\theta}\) does not provide a good estimate for \(\theta\), thus multiple samples will be required to produce estimates for model hyperparameters. It is also clear that having more observations at each node of the phylogeny results in better hyperparamenter estimation. therefore it is important to ensure that this is the case for inference on \(\mathbf{T}\) for bat echolocation calls.  

\subsection{The Evolution of Echolocation in Bats}

Given the functional representation of the bat echolocation calls described a set of deterministic basis functions need to be defined. These basis functions should be derived in a methodologically sound while still having an interpretation that reflects the specific dataset being considered. 

An important consideration in the analysis of any dataset is the notion of sampling bias. Even if we accept that that obtaining a representative sample of all the 1000 plus species of bat, is not possible, the small dataset analysed here is inherently imbalanced, as illustrated in Table \ref{tab:dataset}. At all stages of the analysis reasonable steps were taken to mitigate the effects of sampling bias.

The first step in identifying a set of stable, interpretable, independent basis functions involved a Principal Components Analysis of 1000 samples of Echolocation Calls, balanced by both individual and species. These balanced samples randomly selected 4 individuals per species and then  randomly selected one of these individuals call recordings. Over all these samples it was found that the first 6 principal components generally accounted for 85\% of the variation in the sample, with the \(6^{th}\) component accounting for 3\% of the variation. The first and second Principal components were consistent across samples and so the mean of all these sampled components were identified as the first and second overall principal components. 

The remaining components were then grouped by the frequency with the absolute maximum weight. Grouping components in this way means that each component can be interpreted as the frequency with maximum energy.
In this way 6 overall principal components were identified. These principal components were then passed through the CuBICA algorithm and independent basis functions for the bats echolocation calls identified. These basis functions were then passed on for analysis alongside the phylogenetic Ornstein-Uhlenbeck process.

We are interested in the evolution of echolocation in bats to a species level, thus it is important to consider function valued traits that are representative of the species. Thus we consider estimates of the species level traits by taking balsanced estimates of the mean trait for that species. These mean estimates took one calls from 4 individuals. 4 such estimates were taken to form a sample of traits for type II MLE. 1000 samples were taken and hyperparameters estimated for each basis function. The distribution of these estimates were used to create hyperparameter estimates.

\begin{table}[ht]
	\tbl{\(\hat{\theta}\) Distribution}
	{
		\begin{tabular}{@{}lccccccc@{}} \toprule
			Basis: & Bias & 1 & 2 & 3 & 4 & 5 & 6 \\ 
			\colrule
			\(\sigma_p \)  & mean (std dev) & mean (std dev) & mean (std dev) &&&&  \\
			\(\ell\)  & mean (std dev) & mean (std dev) & mean (std dev) &&&& \\
			\(\sigma_n\)  & mean (std dev) & mean (std dev) & mean (std dev) &&&&  \\
			
			\botrule
		\end{tabular}
	}
	\label{tab::bat_results}
\end{table}

\begin{table}[ht]
	\tbl{\(\hat{\theta}\) Distribution}
	{
		\begin{tabular}{@{}cccc@{}} \toprule
			Basis & \(\sigma_p\) & \(\ell\) & \(\sigma_n\) \\
			\colrule
			1 & 0.35 (0.008) & 224.5 (39.15) & 0.08 (0.007) \\
			2 & 0.51 (0.015) & 287.9 (44.76) & 0.09 (0.009) \\
			3 & 0.42 (0.021)& 204.2 (30.39) & 0.08 (0.009) \\
			4 & 0.59 (0.019) & 390.1 (74.44) & 0.14 (0.013) \\
			5 & 0.51 (0.010) & 339.8 (42.81) & 0.08 (0.008) \\
			6 & 0.48 (0.013) & 148.6 (23.29) & 0.09 (0.011) \\
			\botrule
		\end{tabular}
	}
	\label{tab::bat_results_option2}
\end{table}

Ancestral reconstructions were performed by taking the sample mean for each species and obtaining posterior distributions based on these mean estimates.

\section{Conclusions and Future Work}

Demonstrated an application of phylogenetic Gaussian processes for a dataset. Higlighted issues with balanced samples, how to leverage the dataset to account for this. 

Identified features of calls, corresponding to frequencies. Demonstrated a possible evolutionary dynamic. 

Performed ancestral reconstructions which suggest...

Early stages of this work, there are some extensions to be made. Spectrograms as function valued traits, account for temporal variation in call.

consider other representations.

Consider ways of extending these models to more realistic models of evolution, relax assumptions.

When methodology has bee resolved extend to a larger dataset. EchoBank is there. Hopefully contribute to the understanding of the evolution of echolocation in bats.

\bibliographystyle{ws-rv-van}
\bibliography{../BatBiblio}

%\blankpage
%\printindex[aindx]                 % to print author index
%\printindex                         % to print subject index

\end{document} 